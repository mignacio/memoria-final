\chapter{Introducción específica} % Main chapter title

\label{Chapter2}

En este capítulo se explica en detalle el funcionamiento de un motor de combustión interna y se describe el sistema de desarrollado, sus partes y la interacción entre ellas. Al final enlista las historias de usuario y requerimientos del proyecto.

\section{Funcionamiento de un motor de combustión interna}

Un motor de combustión interna, o motor a explosión, es un tipo de máquina que obtiene energía mecánica a través de una reacción química conocida como combustión. La combustión es la oxidación rápida de materiales llamados combustibles en presencia de oxígeno, al que se le llama comburente. El resultado de esta reacción, cuando la reacción es completa, es dióxido de carbono, agua, y energía liberada en forma de calor y expansión gaseosa. Es la energía liberada en forma de expansión que es convertida en energía mecánica, y la energía liberada en forma de calor es disipada a la atmósfera.

En una combustión completa la totalidad del combustible es oxidado y entonces su eficiencia es máxima. Para que una combustión sea completa se tiene que cumplir la condición de que la proporción por peso de aire y combustible sea igual a su proporción estequiométrica. En un motor de gasolina esta proporción es de 14,7 partes de aire por 1 parte de combustible.

A esta proporción se la llama factor lambda y se designa con la letra griega \lambda. Entonces para poder operar un motor de forma eficiente, se debe controlar la cantidad de combustible utilizado en cada proceso de combustión según el factor lambda

\section{Descripción general del sistema}

Describir las partes que componen al sistema y su funcionamiento.

\section{Historias de usuario y requerimientos}

El desarrollo del Trabajo Final fue basado en las historias de usuario y los requerimientos fijados durante la etapa de planificación del proyecto.

\subsection{Historias de usuario}

En la planificación del proyecto, se obtuvieron historias de usuario según un cliente interesado en instalar este dispositivo en su vehículo antiguo. Las historias compiladas son las siguientes:

\begin{itemize}
\item Como cliente quiero poder visualizar en una pantalla la ifnromación de los sensores en tiempo real en un mismo gráfico lineal para poder ver la evolución de las variables de los sensores.
\item Como cliente quiero poder definir la escala del eje temporal del gráfico lineal, y elegir entre cinco valores distintos. (1, 10, 30, 60 y 120 minutos por división) para poder ver con detalle fenómenos de corta y larga duración
\item Como cliente quiero poder elegir qué sensores deben ser mostrados en pantalla y cuáles no, para poder darle prioridad a la información que es más relevante en ese momento.
\item Como cliente quiero poder definir niveles para cada sensor individualmente y que suene una alarma cuando algún valor supere dicho nivel para ser notificado al instante de que algo no funciona adecuadamente.
\item Como cliente quiero poder descargar toda la información recolectada en una tarjeta SD o pendrive, para poder analizarla en otro momento.
\item Como cliente quiero que el sistema conmience a funcionar cuando se le da contacto al vehículo con la llave de ignición para poder ver en pantalla como funciona el motor durante el arranque.
\end{itemize}

\subsection{Requerimientos}

A partir de las historias de usuario listadas en la sección  se crearon los requerimientos del proyecto detallados más adelante. Los requisitos se divieron en cuatro grupos: Requerimientos generales del proyecto; requerimientos de la interfáz gráfica; requerimientos de la parte adquisidora de datos y requerimientos de la comunicación entre las partes.

Los requerimientos generales del proyecto son:
\begin{itemize}
\item REQ-GEN-001: Todo el código fuente del proyecto será almacenado bajo un sistema de control de versiones GIT.
\item REQ-GEN-002: La documentación del código fuente del sfotware embebidos será llevada a cabo en los comentarios, siguiendo el formato de Doxygen.
\item REQ-GEN-003: La documentación del software para la interfaz gráfica también será llevada a cabo en los comentarios. El formato será elegido por el responsable del proyecto.
\end{itemize}

Requerimientos de la interfáz gráfica:
\begin{itemize}
\item REQ-GUI-001: La interfaz gráfica deberá poder mostrar figuras con la información de todos los sensores a la vez.
\item REQ-GUI-002: El usuario tiene que poder elegir qué sensores ver al mismo tiempo y cuáles no desea ver.
\item REQ-GUI-003: El usuario tiene que poder definir alarmas por valor máximo, para cada una de las variables.
\item REQ-GUI-004: Las alarmas serán sonores y visuales. El estilo de las alarmas será definido por el cliente durante el proceso de desarrollo de la interfáz gráfica.
\end{itemize}

Requerimientos de la parte adquisidora:
\begin{itemize}
\item REQ-ADQ-001: El sistema tiene que adquirir la temperatura de los gases de admisión y escape, con un rango de temperatura entre 0\degree C y 400 \degree C y con una resolución menor igual a 0,5\degree C. Con una tasa de muestreo mayor o igual a 1hz.
\item REQ-ADQ-002: El sistema tiene que adquirir la temperatura del aceite del motor, con un rango de temperatura entre 0\degree C y 400\degree C y una resolución menor igual a 0,5\degree C. Con una tasa de muestreo mayor igual a 1hz.
\item REQ-ADQ-003: El sistema tiene que adquirir la velocidad de giro del motor, en un rango entre 0 y 20.000 revoluciones por minuto, con una resolución menor igual a 500 r.p.m. Con una tasa de muestreo mayor igual a 5hz.
\item REQ-ADQ-004: El sistema tiene que adquirir la proporción de oxígeno en los gases de escape llamada lambda, con un rango de 0 a 2 y una resolución menor igual a 0,1 lambda.
\item REQ-ADQ-005: El sistema tiene que adquirir la presión de aceite del motor, con un rango de 0 a 100 psi y con una resolución menor igual a 1 psi.
\item REQ-ADQ-006: El sistema debe comenzar a transmitir a la interfaz gráfica la información obtenida en un tiempo no mayor a 1 segundo transcurrido el proceso de adquisición.
\end{itemize}

Requerimientos de la comunicación entre las partes del sistema:
\begin{itemize}
\item REQ-COMM-001: Se permitirá que se pierda hasta 1 paquete de cada 100 paquetes transmitidos.
\end{itemize}