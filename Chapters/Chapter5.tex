\chapter{Conclusiones}

\label{Chapter5}

\section{Resultados obtenidos}

Al finalizar el trabajo, se llegó al desarrollo y fabricación de un primer prototipo que cumplió con los requisitos planteados en la etapa de planificación. Para poder llegar a este resultado fue necesario aplicar conocimientos adquiridos durante el cursado de las materias de la especialización, tales como:

\begin{itemize}
\item{Ingeniería de software:} se utilizó el conocimiento adquirido en la materia, para seleccionar las arquitecturas de software, utilizadas en el desarrollo del firmware de la parte adquisidora, y el software de la interfaz gráfica.

\item{Protocolos de comunicación para sistemas embebidos}: se aplicó el contenido de la materia para desarollar el código para configurar y luego transmitir la información adquirida a través del módulo \textit{Bluetooth} HM-10. También fue necesario desarrollar un módulo de software para comunicar la EDU-CIAA con los circuitos integrados de medición de termocuplas por protocolo SPI.

\item{Sistemas operativos de tiempo real I y II:} Para el desarrollo de el firmware de la parte adquisidora, tuvieron que aplicarse los conceptos de separación de tareas, concurrencia, colas y manejo de interrupciones en un sistema operativo de tiempo real.

\item{Diseño de circuitos impresos:} en el cursado de la materia se desarrollo un poncho de EDU-CIAA para uno de los circuitos de medición. Luego, dicho proyecto fue expandido para comprender el circuito entero del dispositivo.
\end{itemize}

\section{Trabajo futuro}
Con el objetivo de continuar el desarrollo para llegar a un dispositivo final que pueda ser comerciable, se identificó que se debe continuar trabajando en los siguientes ítems:

\begin{itemize}
\item Durante la etapa de ensayos fue muy difícil leer la información desde la pantalla LCD de 5 pulgadas, la interfaz gráfica deberá ser optimizada para que pueda ser empleada en pantallas de tamaños reducidos.

\item Para poder proteger a la parte adquisidora de las condiciones que existen dentro del habitáculo del motor, se tendrá que desarrollar o seleccionar un gabinete para resguardarla y evitar posibles daños.

\item Desarrollar un sistema de montaje para poder anclar al gabinete al habitáculo del motor.

\item Mejorar el circuito agregando protecciones contra descargas electroestáticas y radiación electromagnética.

\item Investigar otros métodos de comunicación, especialmente que permitan mayor ancho de banda, para poder aumentar la cantidad de sensores o la tasa de muestreo.

\end{itemize}



