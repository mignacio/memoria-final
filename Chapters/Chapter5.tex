\chapter{Conclusiones}

\label{Chapter5}

En este capítulo, se describen las conclusiones alcanzadas al finalizar el proyecto y se listan los pasos a seguir para llevar el prototipo terminado a un producto terminado al mercado.

\section{Conclusiones generales}

Al finalizar el proyecto, se demostró que fue posible desarrollar y fabricar un prototipo que cumpla con los requisitos planteados. En el transcurso tuvieron que superarse desafíos y resolver problemas. Algunos de ellos fueron, la selección de reemplazo de componentes que estaban en falta, y el desarrollo de una interfaz gráfica de usuario sin experiencia previa.  Durante la etapa de pruebas se encontraron falencias en el diseño principal, como por ejemplo, la pantalla era muy pequeña a la hora de leer la información a distancia o en ángulos muy pronunciados. También resultó difícil realizar algunas de las pruebas, sobre todo a la hora de conectar los sensores al motor, ya que los conectores seleccionados no eran los óptimos para la aplicación. Estas conexiones débiles, en ocasiones generaron lecturas incorrectas causadas por desconecciones imprevistas. Con base en estos, se logró obtener suficiente conocimiento y experiencia para poder continuar con el proyecto, y aplicarlo para avanzar del prototipo a un producto final.

\section{Trabajo futuro}
Con el objetivo de continuar el desarrollo para llegar a un dispositivo final que pueda ser comercial, se identificó que se debe continuar con las mejoras en los siguientes ítems:

\begin{itemize}
\item Optimizar la interfaz gráfica para pantallas de tamaño reducido. El tamaño de la pantalla utilizado resultó muy pequeño cuando se realizaron los primeros ensayos, y dificultó leer la información desde una distancia moderada.

\item Desarrollar o seleccionar un gabinete para poder proteger a la parte adquisidora de las condiciones que existen dentro del habitáculo del motor, y así evitar posibles daños.

\item Desarrollar un sistema de montaje para poder anclar el gabinete al habitáculo del motor.

\item Reemplazar las borneras con tornillos por conectores aptos para trabajar en condiciones de alta temperatura y vibración.

\item Desarrollar un arnés de conexión entre los sensores y el circuito impreso.

\item Mejorar el circuito con el agregado de protecciones contra descargas electroestáticas y radiación electromagnética.

\item Durante el desarrollo el integrado LM9044 fue descontinuado, por lo que será necesario encontrar un producto alternativo o diseñar un amplificador específico para sonda Lambda.

\item Investigar otros métodos de comunicación, especialmente que permitan mayor ancho de banda, para poder aumentar la cantidad de sensores o la tasa de muestreo.

\end{itemize}



