\chapter{Conclusiones}

\label{Chapter5}

Al finalizar del trabajo, se demostró que fue posible desarrollar y fabricar un prototipo que cumpla con todos los requisitos planteados durante la fase de planificación del proyecto. En el transcurso tuvieron que superarse desafíos y resolver problemas. Como por ejemplo, la selección de reemplazo de componentes que estaban en falta, y el desarrollo de una interfaz gráfica de usuario, sin experiencia previa. Gracias a eso se logró obtener suficiente conocimiento y experiencia para poder continuar con el trabajo, y aplicarlo para avanzar del prototipo a un producto final.

\section{Trabajo futuro}
Con el objetivo de continuar el desarrollo para llegar a un dispositivo final que pueda ser comerciable, se identificó que se debe continuar trabajando en los siguientes ítems:

\begin{itemize}
\item Durante la etapa de ensayos fue muy difícil leer la información desde la pantalla LCD de 5 pulgadas, la interfaz gráfica deberá ser optimizada para que pueda ser empleada en pantallas de tamaños reducidos.

\item Para poder proteger a la parte adquisidora de las condiciones que existen dentro del habitáculo del motor, se tendrá que desarrollar o seleccionar un gabinete para resguardarla y evitar posibles daños.

\item Reemplazar las borneras con tornillos por conectores aptos para trabajar en condiciones de alta temperatura y vibración.

\item Desarrollar un sistema de montaje para poder anclar al gabinete al habitáculo del motor.

\item Desarrollar un arnés de conexión entre los sensores y el circuito impreso.

\item Mejorar el circuito agregando protecciones contra descargas electroestáticas y radiación electromagnética.

\item Investigar otros métodos de comunicación, especialmente que permitan mayor ancho de banda, para poder aumentar la cantidad de sensores o la tasa de muestreo.

\end{itemize}



