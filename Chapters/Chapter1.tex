% Chapter 1

\chapter{Introducción general} % Main chapter title

Este capitulo explica los motivos detrás de la decisión de realizar este trabajo final. Realiza una comparación entre el dispositivo desarrollado y productos de características similares que están disponibles en el mercado. Y finalmente enumera los alcances y objetivos del trabajo final.

\label{Chapter1} % For referencing the chapter elsewhere, use \ref{Chapter1} 
\label{IntroGeneral}

%----------------------------------------------------------------------------------------

% Define some commands to keep the formatting separated from the content 
\newcommand{\keyword}[1]{\textbf{#1}}
\newcommand{\tabhead}[1]{\textbf{#1}}
\newcommand{\code}[1]{\texttt{#1}}
\newcommand{\file}[1]{\texttt{\bfseries#1}}
\newcommand{\option}[1]{\texttt{\itshape#1}}
\newcommand{\grados}{$^{\circ}$}

%----------------------------------------------------------------------------------------

%\section{Introducción}

%----------------------------------------------------------------------------------------
\section{Motivación}

Desde el año 1991 en adelante, todos los fabricantes de vehículos de combustión interna están obligados a incluir en sus vehículos un sistema electrónico de diagnóstico. Este sistema, conocido por sus siglas en inglés como OBD (On-Board Diagnostics), realiza las tareas de muestrear sensores que están conectados físicamente sobre el motor y alertar al conductor, a través de un indicador en el tablero, cuando el motor no está funcionando dentro de los parámetros de operación. También mantiene un registro interno de los fallos ocurridos durante la vida del mismo, para luego ser descargado por el mecánico encargado de realizar tareas de reparación o puesta a punto, y así facilitar su trabajo.
Actualmente existen grupos de entusiastas y coleccionistas que poseen vehículos fabricados antes de que el sistema OBD se haga obligatorio. Y por esa razón, no tienen la posibilidad de hacer un monitoreo del funcionamiento del motor de su vehículo. Tampoco es posible mantener un registro de si hubo eventos de fallas o momentos de operación fuera de rango, información útil para su mantenimiento preventivo.
Por esto es que se tomó la decisión de desarrollar para este Trabajo Final, un dispositivo que cumpla las mismas funciones que un OBD, pero para vehículos antiguos que no tienen instalado dicho sistema de fábrica.

\section{Estado del arte}

Comparación del sistema con productos similares del mercado.

\section{Alcance y objetivos}

Explicar que se pretende obtener finalizado el desarrollo.


